\documentclass[10pt]{revtex4}

\usepackage{amsmath}
\usepackage{amssymb}
\usepackage{graphicx}

\begin{document}

Our goal, broadly speaking, is to duplicate the ``branching process" approximation of Desai and Fisher (2007) incorporating density-dependence, which is a simple way to model the evolution of cooperation.
Cooperation becomes a better strategy as more individuals cooperate.

When the number of mutant individuals $n$ is small enough compared to the population size $N$, it does not yet affect the mean fitness.
$n$ can be described by the master equation:
\begin{equation}
P(n,t+dt) = \delta(n+1)P(n+1,t)dt + \beta(n-1)P(n-1,t)dt + (1-(\delta + \beta)dt n)P(n,t),
\end{equation}
where $\delta$ is the death rate and $\beta$ the birth rate.
Essentially, the first term is the probability that at the previous time step there were $n+1$ individuals, one of whom died, the second is the probability that there were $n-1$ individuals, one of whom gave birth, and the last term is the probability that nothing happened.
Without loss of (much) generality, let $\delta = 1$ and $\beta = 1+s$, so that the time scale is essentially set by the death rate.
Then we can rewrite the above as a differential equation for $P(n,t) = P_n$:
\begin{equation}
\partial_t P_n = (n+1)P_{n+1} + (1+s)(n-1)P_{n-1} - (2+s) nP_n.
\end{equation}
(We subtracted out a factor of $P(n,t)$ and took the limit as $dt \to 0$.)

We would like to remove the dependence on $n$.
We do so by introducing the \emph{generating function} $G(z,t) = \sum_n P_n z^n$.
Multiplying through by $z^n$ and summing yields
\begin{equation}
\partial_t  \sum_n P_n z^n =  \sum_n (n+1)P_{n+1} z^n + (1+s) \sum_n (n-1)P_{n-1}z^n - (2+s)\sum_n nP_n z^n.
\end{equation}
Now we need to think carefully about each term in order to formulate it as a differential equation in $G$.
The LHS becomes $\partial_t G$.
The first RHS term becomes $\sum_n (n+1)P_{n+1} z^n = \sum_n n P_{n} z^{n-1}$ (by relabeling elements), which is just $\partial_z G$.
The second term becomes $(1+s) \sum_n (n-1)P_{n-1}z^n = (1+s) \sum_n n P_n z^{n+1}$ (again by relabeling elements), which becomes $(1+s)z^2 \sum_n n P_n z^{n-1} = (1+s) z^2 \partial_z G$.
The last term, by similar reasoning, becomes $-(2+s) z \partial_z G$.

We thus have the PDE
\begin{equation}
\partial_t G = ((1+s)z^2 - (2+s)z + 1)\partial_z G.
\end{equation}
Letting $z = z(t)$, this equation has the form
\begin{equation}
\frac{dG}{dt} = \partial_z G \frac{dz}{dt} + \partial_t G = 0,
\end{equation}
provided that $\frac{dz}{dt} = -((1+s)z^2 - (2+s)z + 1)$.
We will proceed using the ``method of characteristics", solving the above ODE for $\frac{dz}{dt}$ that allows $G$ to be constant along this curve.
(This is where the frequency-dependent version is going to break down, but there may be another solution.)
In other words, we will try to solve the system of equations:
\begin{align}
\frac{dz}{dt} &= -((1+s)z^2 - (2+s)z + 1), \nonumber \\
\frac{dt}{dt} &= 1, \nonumber \\
\frac{dG}{dt} &= 0 \nonumber \\
\end{align}
given the initial conditions
\begin{align}
z(0) &= z_0, \nonumber \\
t(0) &= 0, \nonumber \\
G(0) = G(z(0),0) &= z_0. \nonumber \\
\end{align}
The condition on $G(z(0),0)$ corresponds to specifying that there be one mutant individual at time $0$: for then $G(z,t) = \sum_n P_n(t) z(t)^n = \sum_n P_n(0) z(0)^n = 1 \times z(0)$.
We also have the condition $G(1,t) = \sum_n P_n(t) = 1$, since $P_n$ is a probability mass.

The ODE for $z$ is readily solved:
\begin{equation}
-\int_{z_0}^z \frac{dz^\prime}{((1+s)z^{\prime 2} - (2+s)z^\prime + 1} = \int_0^t dt^\prime = t = -\frac{1}{s}\log\frac{(z-1)((1+s)z_0 - 1)}{((1+s)z-1)(z_0-1)},
\end{equation}
which gives
\begin{equation}
z_0 = \frac{(z-1)-((1+s) z - 1) e^{-st} }{(1+s)(z-1) - ((1+s) z - 1) e^{-st}}.
\end{equation}
We could solve for $z$ instead if we wanted, but $z_0$ is what we really need.
The ODE for $G$ yields $G(t) = \mathrm{constant}$, so the time dependence of $G$ can come in only through $z(t)$.
Effectively, this means $G(z(t),t) = z(t)$.
Combining these yields our generating function
\begin{equation}
G(z,t) = \frac{(z-1)-((1+s))z -1) e^{-st} }{(1+s)(z-1) - ((1+s) z - 1) e^{-st}}.
\end{equation}
Note that multiplying through by $e^{st}$ yields Desai and Fisher's equation 9.
If we had left $\beta$ and $\delta$ explicit, we would have
\begin{equation}
G(z,t) = \frac{\delta (z-1)-(\beta z - \delta) e^{(\delta - \beta) t} }{\beta(z-1) - (\beta z - \delta) e^{(\delta - \beta) t}}.
\end{equation}
From here, properties like fixation probability and establishment time can be obtained.

Let us see what changes when we introduce density dependence.
The simplest possible model is arguably $\beta = 1+ns$.
The more cooperators there are, the more effectively they gather resources and therefore the more quickly they can reproduce.
The equation for $P_n$ becomes
\begin{equation}
\partial_t P_n = (n+1)P_{n+1} + (1+ns)(n-1)P_{n-1} - (2+ns) nP_n.
\end{equation}
Now, again, we need to think carefully as we introduce the generating function:
\begin{equation}
\partial_t \sum_n P_n z^n = \sum_n (n+1)P_{n+1}z^n + \sum_n (1+ns)(n-1)P_{n-1}z^n - \sum_n (2+ns) nP_n z^n.
\end{equation}
Getting rid of the $n^2$ terms within the sums will require that we have some $\partial_z^2$ terms floating around.
Let's see where.

As before, the LHS becomes $\partial_t G$.
The first RHS term becomes $\partial_z G$.
The second RHS term is more interesting. We have
\begin{align}
\sum_n (1+ns)(n-1)P_{n-1}z^n &= \sum_n (n(n-1)s + (n-1))P_{n-1}z^n \nonumber \\
&= s\sum_n n(n-1)P_{n-1}z^n + \sum_n (n-1)P_{n-1}z^n \nonumber \\
&= s\sum_n (n+1)nP_n z^{n+1} + \sum_n n P_n z^{n+1} \nonumber \\
&= sz^2\sum_n (n+1)nP_n z^{n-1} + z^2 \sum_n n P_n z^{n-1} \nonumber \\
&= sz^2\partial_z^2 (zG) + z^2 \partial_z G \nonumber \\
&= sz^2 \partial_z (G + z\partial_z G) + z^2 \partial_z G \nonumber \\
&= sz^2 (z\partial_z^2 G + 2\partial_z G) + z^2\partial_z G \nonumber \\
&= sz^3 \partial_z^2 G + (2sz^2 + z^2)\partial_z G.
\end{align}
Lastly, that third RHS term becomes
\begin{align}
\sum_n (2+ns) nP_n z^n &=& 2\sum_n nP_n z^n + s\sum_n n^2 P_n z^n \nonumber \\
&= 2z \sum_n nP_n z^{n-1} + s\sum_n n(n-1) P_n z^n + s\sum_n nP_n z^n \nonumber \\
&= 2z \partial_z G + sz^2\sum_n n(n-1) P_n z^{n-2} + sz\sum_n nP_n z^{n-1} \nonumber \\
&= sz^2 \partial_z^2 G + (2z+sz)\partial_z G.
\end{align}
Putting it all together, we get
\begin{equation}
\partial_t G = sz^2(z-1) \partial_z^2 G + (z^2(2s+1) - z(2+s) + 1)\partial_z G.
\end{equation}
There might be a way to solve this, since at least the terms are nice.
This is a parabolic PDE, something similar to the heat or diffusion equation.
We might unfortunately have to ``guess" the right form for the solution, thought we might be able to use characteristics somehow.


\end{document}