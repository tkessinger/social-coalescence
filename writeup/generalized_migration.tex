\documentclass[rmp]{revtex4}
\usepackage[english]{babel}
\usepackage{amsmath}
\usepackage{amssymb}
\usepackage{amsfonts}

\begin{document}

We are using Wakeley's (1998) formulation to calculate the expected coalescence times and transition rates in a subdivided population under a \emph{general} $\Lambda$-coalescent model, with $D$ demes and migration rate $m$.
(Within-deme population size $N$ will not be important. See below.)
Let $t(n_1,n_2,\ldots n_n)$ represent the total length of the genealogy of a sample with configuration $(n_1,n_2,\ldots n_n)$, where $n_i$ is the number of demes from which $i$ sequences are sampled.
We will consider transitions from one configuration to another and use the recursions

\begin{equation}
\mathrm{E} \left[ T_i \right] = \frac{n}{P_{i*}} + \sum_{j=1,j \neq i}^R \frac{P_{ij}}{P_{i*}} \mathrm{E} \left[ T_j \right]
\end{equation}
and
\begin{equation}
\mathrm{Var} \left[ T_i \right] = \left( \frac{n}{P_{i*}} \right)^2 + \sum_{j=1,j \neq i}^R \frac{P_{ij}}{P_{i*}} (\mathrm{Var} \left[ T_i \right] + \mathrm{E} \left[ T_j \right]^2) - \left( \sum_{j=1,j \neq i}^R \frac{P_{ij}}{P_{i*}} \mathrm{E} \left[ T_j \right] \right)^2.
\end{equation}

Here, $i$ and $j$ are any two states of the system, $P_{ij}$ is the transition rate therebetween, $T_i$ is the total length of the branches in such a genealogy, $n$ is the sample size (which changes: perhaps $b$ would be a better way to frame this), and $P_{i*}$ is the sum of all transition rates out of state $i$.

We will keep the following as general as possible, allowing $\lambda_{b,k}$ to be the rate at which $k$ lineages out of $b$ extant ones merge (for the Kingman coalescent, this is simply $\lambda_{b,2} = b(b-1)/N$ and all other $\lambda_{b,k} = 0$ for a haploid population of size $N$).
Notably, these rates will (depending on the particular coalescent model) \emph{not} necessarily depend on $1/N$.

Moving on, we will consider the case of two individuals.
There are three possible states, with their associated transition rates

$(2,0) \to (0,1)$ with rate $2m/(D-1)$,

$(0,1) \to (2,0)$ with rate $2m$,

$(0,1) \to (1,0)$ with rate $\lambda_{2,2}$.

The above formulation for the expectation value gives
\begin{equation}
\mathrm{E} \left[ t(0,1) \right] = \frac{2}{2m + \lambda_{2,2}} + \frac{2m}{2m + \lambda_{2,2}} \mathrm{E} \left[ t(2,0) \right] + \frac{\lambda_{2,2}}{2m + \lambda_{2,2}} \mathrm{E} \left[ t(1,0) \right]
\end{equation}
and
\begin{equation}
\mathrm{E} \left[ t(2,0) \right] = \frac{2}{2m/(D-1)} + \mathrm{E} \left[ t(0,1) \right].
\end{equation}
In the first expression, the last term $\to 0$ because $\mathrm{E} \left[ t(1,0) \right] = 0$.
Wakeley uses $\lambda_{2,2} = 1/2N$ (diploid population) to obtain $\mathrm{E} \left[ t(0,1) \right] = 4ND$ and $\mathrm{E} \left[ t(2,0) \right] = 4ND + (D-1)/m$.
The generalized coalescent gives $2D/\lambda$ and $2D/\lambda + (D-1)/m$.

Next we will consider a slightly more involved example, with three individuals.

The transition rates due to migration are

$(3,0,0) \to (1,1,0)$ with rate $6m/(D-1)$ (any of the three individuals can migrate to either of the other individuals' demes),

$(1,1,0) \to (3,0,0)$ with rate $2m(D-2)/(D-1)$ (one of the two $n_2$ individuals must migrate to one of the $D-2$ empty demes),

$(1,1,0) \to (0,0,1)$ with rate $m/(D-1)$ (the $n_1$ individual must migrate to the $n_2$ deme),

$(0,0,1) \to (1,1,0)$ with rate $3m$ (any individual can migrate out),

$(2,0,0) \to (0,1,0)$ with rate $2m/(D-1)$,

$(0,1,0) \to (2,0,0)$ with rate $2m$.

The transition rates due to coalescence are:

$(1,1,0) \to (2,0,0)$ with rate $\lambda_{2,2}$,

$(0,0,1) \to (0,1,0)$ with rate $3\lambda_{3,2}$ (any of the three pairs can coalesce),

$(0,1,0) \to (1,0,0)$ with rate $\lambda_{2,2}$,

$(0,0,1) \to (1,0,0)$ with rate $\lambda_{3,3}$.

The last and second to last transition terms are the ones that will differ from Wakeley's results.
We end up with a $6 \times 6$ transition rate matrix, but we can start solving ``from the end", so to speak, and use $\mathrm{E} \left[ t(1,0,0) \right] = 0$:

\begin{align}
\mathrm{E} \left[ t(0,1,0) \right] &= \frac{1}{2m + \lambda_{2,2}} (2 + 2m \mathrm{E} \left[ t(2,0,0) \right]), \\
\mathrm{E} \left[ t(2,0,0) \right] &= \frac{1}{2m/(D-1)} (2 + 2m/(D-1) \mathrm{E} \left[ t(0,1,0) \right] ), \\
\mathrm{E} \left[ t(0,0,1) \right] &= \frac{1}{3m + 3\lambda_{3,2} + \lambda_{3,3}} (3 +3m \mathrm{E} \left[ t(1,1,0) \right] + 3\lambda_{3,2} \mathrm{E} \left[ t(0,1,0) \right]),\\
\mathrm{E} \left[ t(1,1,0) \right] &= \frac{1}{2m(D-2)/(D-1) + m/(D-1) + \lambda_{2,2}} (3 + 2m(D-2)/(D-1) \mathrm{E} \left[ t(3,0,0) \right] \\
&+ m/(D-1) \mathrm{E} \left[ t(0,0,1) \right] + \lambda_{2,2} \mathrm{E} \left[ t(2,0,0) \right]), \\
\mathrm{E} \left[ t(3,0,0) \right] &= \frac{1}{6m/(D-1)} (3 + 6m/(D-1) \mathrm{E} \left[ t(1,1,0) \right]).
\end{align}

As expected, the top two equations look familiar, yielding 
\begin{align}
\mathrm{E} \left[ t(0,1,0) \right] & = 2D/\lambda_{2,2}, \\
\mathrm{E} \left[ t(2,0,0) \right] & = 2D/\lambda_{2,2} + (D-1)/m.
\end{align}

Suppose that $\lambda_{3,2} = \lambda_{2,2} = \lambda$ and $\lambda_{3,3} = 0$.
Then we have
\begin{align}
\mathrm{E} \left[ t(0,0,1) \right] &= 3D/\lambda,\\
\mathrm{E} \left[ t(1,1,0) \right] &= 3D/\lambda +(D-1)/m, \\
\mathrm{E} \left[ t(3,0,0) \right] &= 3D/\lambda  + 3(D-1)/2m.
\end{align}
These are consistent with eqns. 13-15 from Wakeley.
Suppose we allow the $\lambda_{b,k}$ to take on more general values. Then the expressions become much uglier:
\begin{align}
\mathrm{E} \left[ t(0,0,1) \right] &= (3 D ((D-1) \lambda_{2,2} (2 \lambda_{3,2} + \lambda_{2,2}) + (2 \lambda_{3,2} \nonumber \\
& + (3D-2) \lambda_{2,2}) m)) / (\lambda_{2,2} ((D-1) (\lambda_{3,3} \nonumber \\
& + 3 \lambda_{3,2}) \lambda_{2,2} + (\lambda_{3,3} + 3 (\lambda_{3,2} + (D-1) \lambda_{2,2})) m)), \\
\mathrm{E} \left[ t(1,1,0) \right] &= (((D-1)^2 (\lambda_{3,3} + 3 \lambda_{3,2}) \lambda_{2,2}^2 \nonumber \\
& + (D-1) \lambda_{2,2} ((3D-1) (\lambda_{3,3} + 3 \lambda_{3,2}) + 
       3 (D-1) \lambda_{2,2}) m + 3 D (2 \lambda_{3,2} + (3D-2) \lambda_{2,2}) m^2) /
   \lambda_{2,2} m ((D-1) (\lambda_{3,3} \nonumber \\
& + 3 \lambda_{3,2}) \lambda_{2,2} + (\lambda_{3,3} + 3 (\lambda_{3,2} + (D-1) \lambda_{2,2})) m), \\
\mathrm{E} \left[ t(3,0,0) \right] &= ((3 (D-1)^2 (\lambda_{3,3} + 3 \lambda_{3,2}) \lambda_{2,2}^2 \nonumber \\
& + 
    3 (D-1) \lambda_{2,2} ((2D+1) (\lambda_{3,3} + 3 \lambda_{3,2}) + 3 (D-1) \lambda_{2,2}) m + 
    6 D (2 \lambda_{3,2} + (3D-2) \lambda_{2,2}) m^2) / (
   2 \lambda_{2,2} m ((D-1) (\lambda_{3,3} \nonumber \\ 
& + 3 \lambda_{3,2}) \lambda_{2,2} + (\lambda_{3,3} + 3 (\lambda_{3,2} + (D-1) \lambda_{2,2})) m)).
\end{align}
There might be a useful way to simplify this, but for now, let's consider a more specific case, the Bolthausen-Sznitman coalescent, where $\lambda_{b,k} = \frac{(k-2)!(b-k)!}{(b-1)!}$ times a rate parameter (which we'll label simply $\lambda$).
This implies $\lambda_{2,2} = 1$, $\lambda_{3,2} = 1/2$, and $\lambda_{3,3} = 1/2$ (all times the constant rate parameter $\lambda$).
The above expressions become
\begin{align}
\mathrm{E} \left[ t(0,0,1) \right] &= 3D/\lambda,\\
\mathrm{E} \left[ t(1,1,0) \right] &= 3D/\lambda +(D-1)/m, \\
\mathrm{E} \left[ t(3,0,0) \right] &= 3D/\lambda  + 3(D-1)/2m,
\end{align}
so that there is (sadly) \emph{no change} from the Kingman case.
(To-do: what happens in the case of an intermediate Beta coalescent, say $\alpha = 1.5$?)

It is disconcerting that the coalescent process has essentially no effect on the total branch length, but this should perhaps not be surprising.
Consider the $D=1$ case.
Then the average total branch length on a Kingman tree (in units of $\lambda$) will be $3$ (invert the merger rate of $k(k-1)/2$, times the number of extant branches $k$, ranging from $2$ to $3$), but the average total branch length on a BSC tree will be the same: the average wait time until the first merger event will be $1/2$ (it will be binary with probability $3/4$ and trinary with probability $1/4$), and if it is binary, the wait time until the last merger will be $1$ on average: so we have $1/4(3 \times 1/2) + 3/4(3 \times 1/2 + 2 \times 1) = 3/8+21/8 = 3$.

Marked differences between the BSC and Kingman coalescents will likely manifest only at large $n$, which somewhat dampens the hope of being able to apply this to smaller sample sizes.
The $n=4$ case may still be interesting, however, and the relatively poor correlation between $N$ and coalescence times (often encapsulated via $N_e$) may be worth mentioning.
In fact, we can predict that the BSC dynamics will generally be more conducive to the evolution of cooperation.

The situation becomes slightly different when, instead of considering the total branch length on the tree, we consider only the pairwise coalescence times.
The recursion equations become

\begin{align}
\mathrm{E} \left[ t(0,1,0) \right] &= \frac{1}{2m + \lambda_{2,2}} (1 + 2m \mathrm{E} \left[ t(2,0,0) \right]), \\
\mathrm{E} \left[ t(2,0,0) \right] &= \frac{1}{2m/(D-1)} (1 + 2m/(D-1) \mathrm{E} \left[ t(0,1,0) \right] ), \\
\mathrm{E} \left[ t(0,0,1) \right] &= \frac{1}{3m + 3\lambda_{3,2} + \lambda_{3,3}} (1 +3m \mathrm{E} \left[ t(1,1,0) \right] + 3\lambda_{3,2} \mathrm{E} \left[ t(0,1,0) \right]),\\
\mathrm{E} \left[ t(1,1,0) \right] &= \frac{1}{2m(D-2)/(D-1) + m/(D-1) + \lambda_{2,2}} (1 + 2m(D-2)/(D-1) \mathrm{E} \left[ t(3,0,0) \right] \\
&+ m/(D-1) \mathrm{E} \left[ t(0,0,1) \right] + \lambda_{2,2} \mathrm{E} \left[ t(2,0,0) \right]), \\
\mathrm{E} \left[ t(3,0,0) \right] &= \frac{1}{6m/(D-1)} (1 + 6m/(D-1) \mathrm{E} \left[ t(1,1,0) \right]).
\end{align}

In the Kingman limit, the coalescence times are

\begin{align}
\mathrm{E} \left[ t(0,0,1) \right] &= \frac{1}{6}\left( \frac{8D}{\lambda} + \frac{(D-1)^2}{(D-1)\lambda+Dm} \right),\\
\mathrm{E} \left[ t(1,1,0) \right] &= \frac{1}{6} \left( \frac{8D}{\lambda} + \frac{3(D-1)}{m} - \frac{D-1}{(D-1)\lambda+Dm} \right), \\
\mathrm{E} \left[ t(3,0,0) \right] &= \frac{1}{6}\left( \frac{8D}{\lambda} + \frac{4(D-1)}{m} - \frac{D-1}{(D-1)\lambda+Dm} \right),
\end{align}

which (thankfully) agrees with Van Cleve (2014) under the transformations $D = n$, $\lambda \to 1/N$, $ND = N_T$, $NDm/(D-1) = M$. In the Bolthausen-Sznitman limit, the pair coalescence times thankfully \emph{do} change:

\begin{align}
\mathrm{E} \left[ t(0,0,1) \right] &= \frac{(D-1)(6D-1)\lambda + Dm(8D-3)}{2\lambda(2\lambda(D-1) + m(3D-1))},\\
\mathrm{E} \left[ t(1,1,0) \right] &= \frac{6(D-1)^2\lambda^2 + 5(D-1)(5D-1)\lambda m + 3D(8D-3)m^2}{6\lambda m(2\lambda(D-1) + m(3D-1))},\\
\mathrm{E} \left[ t(3,0,0) \right] &=  \frac{(2(D-1)\lambda + 3Dm)(4(D-1)\lambda + (8D-3)m)}{6\lambda m(2\lambda(D-1) + m(3D-1))}.
\end{align}

These can be partial fraction decomposed into:

\begin{align}
\mathrm{E} \left[ t(0,0,1) \right] &= \frac{D(8D-3)}{2\lambda(3D-1)} + \frac{2D^3-5D^2+4D-1}{2(3D-1)(3Dm-m+2D\lambda - 2\lambda)},\\
\mathrm{E} \left[ t(1,1,0) \right] &= \frac{D(8D-3)}{2\lambda(3D-1)} + \frac{D-1}{2m} + \frac{27D^3-49D^2+27D-5}{6(3D-1)(m(3D-1)+2\lambda(D-1))} + \frac{(D-1)(3D-1)}{2((3D-1)m+2(D-1)\lambda)},\\
\mathrm{E} \left[ t(3,0,0) \right] &= \frac{D(8D-3)}{2\lambda(3D-1)} + \frac{2(D-1)}{3m} + \frac{18D^3-32D^2+17D-3}{3(3D-1)(m(3D-1)+2\lambda(D-1))} + \frac{2(D-1)(3D-1)}{3((3D-1)m+2(D-1)\lambda)}.
\end{align}
(N.B.: there are some slightly nicer versions of these expressions, need to add them in: essentially one combines the last two terms.)

Considering limiting behavior can be useful.
If $m \to \infty$, then most of the terms drop out, and we are left with simply $D(8D-3)/\lambda (6D-2)$ in the BSC case and $4D/3\lambda$ in the Kingman case, for all coalescence times.
A similar result emerges if $\lambda \to 0$.
The reason is that if the ratio $m/\lambda$ is very large, then migration and coalescence are essentially decoupled.
The population appears to be in panmixis, as far as coalescence is concerned.
This immediately suggests that this ratio may be a useful parameter, similar to $D$ defined in Wakeley (1998) and so on.
On the other hand, if $m \to 0$ or $\lambda \to \infty$, then the ``multi-group" coalescence times become $(D-1)/2m$ or $2(D-1)/3m$, respectively; but $\mathrm{E} \left[ t(0,0,1) \right]$ differs, being $(9D-1)/6\lambda$ for the Kingman and $(6d-1)/4\lambda$ for the BSC, a modest change.

$D$ has slightly more interesting effects on the coalescence times.
Consider $\mathrm{E} \left[ t(0,0,1) \right]$.
Large values of $D$ return $4D/3\lambda + D/6(\lambda+m)$ for the Kingman and $4D/3\lambda + D/(9m+6\lambda)$ for the BSC, or slightly lower coalescence times in the BSC.
(That might seem counterintuitive, but recall that this is conditioned on the individuals currently being in the same deme, which, as $D$ becomes large, will not happen often.)
On the other hand, at small $D$ (say $2$), the coalescence time in the Kingman limit is $8/3\lambda$, but it is $13/5\lambda$ in the BSC, so it still turns out not to matter much.
At large $D$ again, the other two coalescence times become $4D/3\lambda + 2D/3m + D/6(\lambda+m)$ in the Kingman and $4D/3\lambda + 2D/3m + 3D/(4\lambda+6m)$ and $4D/3\lambda + 2D/3m + 4D/(2\lambda+3m)$ in the BSC.
As a sanity check, with only one deme, the coalescence times are identical between the BSC and Kingman: I say this is a sanity check because the coalescent of a subsample is itself a coalescent, and $\lambda$ effectively sets the pair coalescence time.
When the number of demes is small, the possibility of having three individuals in the same deme means that pair coalescence times between the Kingman and BSC are comparable: otherwise, Kingman coalescence occurs more quickly.




In the $n=4$ case, the following (new) transition probabilities due to migration arise:

$(0,0,0,1) \to (1,0,1,0)$ with rate $4m$,

$(1,0,1,0) \to (0,0,0,1)$ with rate $m/(D-1)$,

$(1,0,1,0) \to (2,1,0,0)$ with rate $3m(D-2)/(D-1)$,

$(1,0,1,0) \to (0,2,0,0)$ with rate $3m/(D-1)$,

$(2,1,0,0) \to (1,0,1,0)$ with rate $2m/(D-1)$,

$(2,1,0,0) \to (0,2,0,0)$ with rate $2m/(D-1)$,

$(2,1,0,0) \to (4,0,0,0)$ with rate $2m(D-3)/(D-1)$,

$(0,2,0,0) \to (2,1,0,0)$ with rate $4m(D-2)/(D-1)$,

$(0,2,0,0) \to (1,0,1,0)$ with rate $4m/(D-1)$,

$(4,0,0,0) \to (2,1,0,0)$ with rate $12m/(D-1)$.

The following (new) transition probabilities due to coalescence arise:

$(0,0,0,1) \to (0,0,1,0)$ with rate $6\lambda_{4,2}$,

$(0,0,0,1) \to (0,1,0,0)$ with rate $4\lambda_{4,3}$,

$(0,0,0,1) \to (1,0,0,0)$ with rate $\lambda_{4,4}$,

$(1,0,1,0) \to (1,1,0,0)$ with rate $3\lambda_{3,2}$,

$(1,0,1,0) \to (2,0,0,0)$ with rate $\lambda_{3,3}$,

$(0,2,0,0) \to (1,1,0,0)$ with rate $2\lambda_{2,2}$,

$(2,1,0,0) \to (3,0,0,0)$ with rate $\lambda_{2,2}$.

These can be solved by similar methods.

\end{document}